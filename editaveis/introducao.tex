\chapter[Introdução]{Introdução}

    Vivemos na era da tecnologia, a principal engrenagem que move nossa atualidade é a
    informação. A informação é vital pois é dela que a sociedade deriva o conhecimento
    utilizado para crescer. Nos últimos anos a quantidade de dados tem crescido
    exponencialmente com o advento de novas tecnologias e a evolução das mídias sociais.
    Segundo a \textit{International Data Corporation}\footnote{Empresa privada que presta
    consultoria na área de negócios e TI.} (IDC) em 2010, a quantidade de informações
    passou de 1 \textit{zettabyte}\footnote{1 Zettabyte equivale à 1.8 trilhão de
    Gigabytes.}, e em 2011 alcançou a marca de 1.8 \textit{zettabyte}, informações que
    foram capturadas, armazenadas, criadas ao longo dos anos e no fim foram mensuradas
    para calcular o tamanho do espaço digital \cite{gantz2011}.

    Segundo \citeonline{gantz2011} a tendência é que a quantidade de informações dobre a
    cada ano juntamente com o \textit{ratio} de crescimento da informação . De acordo
    com \citeonline{wu2014} 90\% das informações atualmente, foram produzidas nos
    últimos 2 anos, ultrapassando o crescimento da capacidade de armazenamento. Com
    essa grande quantidade de dados e informações, é possível realizar análises, criar
    estratégias de gerenciamento de informações, e integra-las a novos tipos de dados
    juntamente com dados tradicionais, esse processo é o que chamamos de \textit{Big
    Data} \cite{oracle2015}.

    \section{Contextualização}

        Quando pensamos em deslocamento humano nas grandes cidades, logo pensamos no
        conceito de mobilidade urbana, que dê forma bem simples pode ser definido como o
        deslocamento de pessoas dentro do perímetro urbano. Para \citeonline{pontes2011}
        esse conceito vai além de um termo simplesmente quantitativo, e se aproxima mais para
        o qualitativo uma vez que a mobilidade se refere à capacidade que uma pessoa ou grupo
        de indivíduos possuem de se movimentar. A capacidade que está relacionada aos modos
        de transporte em função de uma atividade a ser realizada, é o que torna possível entendermos
        o propósito e definirmos diferentes perfis que atuam nas grandes metrópoles \cite{pontes2011}.

        Com o crescimento rápido das cidades, os planejadores urbanos precisam ter consciência
        dessa realidade, pois enfrentarão inúmeros desafios de trabalho, e o principal deles é o
        planejamento do fluxo urbano nas estradas. Para que eles possam realizar um planejamento
        efetivo é necessário que haja um estudo da mobilidade urbana na área a qual se deseja
        realizar mudanças. Hoje em dia existem inúmeras formas de realizar um mapeamento da
        mobilidade urbana, seja por GPS, aplicativos de smartphone como \textit{Foursquare}\footnote{
        Aplicativo em que o usuário posta o lugar o qual se encontra.}, \textit{Twitter}\footnote{
        Empresa responsável pelo aplicativo de post de mensagens ou twit.}, \textit{Waze}\footnote{
        Aplicativo que mostra o fluxo de transito e o melhor caminho para se chegar ao destino.}
        \cite{tugores2013}.

        Uma fonte de informações que possui o potencial para que possa ser feito esse estudo de dinâmica das
        cidades é os dados da rede de celulares \cite{becker2011}. \citeonline{khan2009} explicam que a
        indústria de telecomunicações cresceu tremendamente nos últimos anos, aumentando o
        número de pessoas que utilizam os serviços de telefonia em geral. No contexto das grandes
        metrópoles, é possível dizer que pelo menos todo cidadão dispõe de um celular. Logo as
        companhias de telefonia dispõem de dados para realizar uma analise de perfil do usuário
        bem como o estudo de mobilidade urbana para realização de um marketing direcionado
        \cite{khan2009}.

        Essas informações que são utilizadas para realizar o mapeamento, que as operadoras estão
        começando a usar, são chamadas de \textit{Call Detail Record}\footnote{\textit{Call Detail
        Record} é sinônimo de \textit{Charging Data Record}} (CDR) ou detalhamento do registro
        de ligação \cite{furletti2013}. De acordo com o manual técnico da \citeonline{3gppts2010}
        a definição de CDR é: \begin{quote}“Uma coleção de informações formatadas sobre os
        serviços (e.g. data da ligação realizada, duração da ligação, quantidade de dados transferidos,
        etc) para uso na realização de cobranças e dados de contabilidade. Para cada uma das partes
        a ser cobrada ou a cobrança de todas as partes referentes a um serviço, um CDR separado
        deve ser gerado, ou seja, mais de um CDR pode ser gerado para um único serviço, por
        exemplo, devido à sua longa duração, ou porque uma das partes ainda falta ser
        cobrada.”\end{quote} Já o conceito sucinto por \citeonline{becker2013} diz que cada CDR
        representar uma atividade de um usuário com referências no espaço (no caso o celular o
        qual a ligação foi realizada) e tempo (no caso quando o usuário realizou a ligação).

        Com o CDR então é possível realizar o mapeamento do fluxo urbano em determinado espaço,
        pois esses dados contém informações referentes as antenas o qual o aparelho está conectado,
        logo a análise será pertinente á verificação da movimentação desses aparelhos ao longo das
        torres de sinais. O problema em utilizar esses dados, reside na problemática do gerenciamento
        de informações pois as mesmas precisam ser reorganizadas e analisadas para que possam
        produzir o conhecimento desejado. Devido a este fato, será utilizado o conceito de
        \textit{Big Data} para fazer o manejamento desses de dados com o objetivo de preparar
        essas informações para serem analisadas.

        Com a definição da atividade de mapeamento urbano, surgiu a necessidade de implantar
        um projeto que tratasse da análise da mobilidade urbana em Brasília utilizando CDR. A
        VISENT é uma empresa que presta serviços de consultoria e desenvolvimento de
        soluções no suporte a processos operacionais e de negócio, com o tratamento de
        registros (como os CDR’s), e responsável pelo projeto \cite{visentsite}. A VISENT
        em parceria com o Instituto Federal de Brasília (IFB), elaborou um outro projeto que
        consiste em realizar uma análise de \textit{benchmark}\footnote{Benchmark é um
        processo de comparação de produtos, serviços e práticas empresariais, e é um
        importante instrumento de gestão das empresas.} de seus clusters, em comparação com
        os \textit{benchmarks} dos \textit{clusters} do IFB. O propósito é chegar a uma
        configuração de \textit{cluster} otimizada e eficiente que realize a análise dos CDR’s
        no menor tempo possível.

    \section{Justificativa}

        A decisão de abordar uma análise de desempenho focado em arquiteturas distribuídas
        (\textit{cluster}) parte da necessidade que projeto possui. A necessidade referente parte
        do seguinte contexto, os clusters que realizam análises sobre os CDR’s tomam um tempo
        significativo para que sejam processadas. Existem inúmeras variáveis referentes ao
        cluster que podem interferir em sua eficiência, como a arquitetura, os softwares
        da análise dos dados, como o processamento de dados é realizado, o hardware utilizado,
        as configurações realizadas, a topologia da rede
        \cite{madhavji2015}.

        Com base na justificativa descrita, o projeto irá visar a realização da montagem de um
        cluster e sua configuração, com o objetivo de receber os CDR’s para que seja feito uma
        replicação da análise feita pela empresa citada. Feito essas atividades, serão realizados
        benchmarks no cluster da alvo para que esses dados venham a ser comparados aos
        benchmarks dos clusters o qual foram realizados as análises. O objetivo é tentar alcançar
        uma configuração de \textit{cluster} e \textit{Big Data}, que realize as mesmas análises,
        em um menor tempo de duração, até chegar em uma configuração otimizada e eficiente.

    \section{Questão de Pesquisa}

        Esse TCC procura responder as seguintes questões de pesquisa:

        \begin{itemize}
            \item É possível replicar o mesmo processamento dos dados no mesmo tempo utilizado em uma arquitetura
                      de hardware diferente?
            \item Com uma arquitetura e configuração do ambiente, é possível reduzir o tempo de processamento
                      dos CDR's em comparação com o tempo de processamento realizado pela empresa alvo?
            \item É possível otimizar essa configuração e arquitetura?
        \end{itemize}

    \section{Objetivos}

        Esta seção define os objetivos geral e específicos referentes ao trabalho.

        \subsection{Objetivo Geral}

            Esse TCC tem como objetivo geral, a elaboração de um \textit{cluster} com uma infraestrutura, arquitetura
            e configurações, utilizando-se de ferramentas como \textit{Hadoop, Impala, Spark, HBase} e conceitos de
            \textit{cluster} e \textit{Big Data} que realizem o processamento dos CDR's no menor tempo possível, em
            comparação com os benchmarks de processamento alvo. O objetivo é surgir com uma infraestrutura,
            arquitetura e configuração de modo que possa ser realizado um comparativo entre os benchmarks tanto
            do \textit{cluster} elaborado quanto do \textit{cluster} alvo.

        \subsection{Objetivos Específicos}

            Os objetivos específicos abordados por esse TCC são:

            \begin{itemize}
                \item Realizar a revisão sistemática sobre o assunto de mobilidade urbana, CDR's, \textit{Big Data, Clusters, Hadoop,
                          MapReduce, Impala, Spark} e de futuras ferramentar que poderão auxiliar na redução do tempo de
                          processamento dos CDR's.
                \item Realizar estudo sobre a infraestrura, arquitetura e configurações necessárias para a elaboração do cluster.
                \item Montar o cluster, e realizar testes básicos para as funcionalidades que serão utilizadas.
                \item Selecionar as métricas que serão utilizadas nos benchmarks comparativos.
                \item Replicar o processamento dos CDR's.
                \item Elaborar configurações, arquiteturas e infraestruturas customizáveis para o processamento de dados no \textit{cluster}.
                \item Comparar os benchmarks obtidos com os benchmarks do processamento alvo e averiguar se o tempo
                          de processamento de dados diminuiu.
                \item Realizar considerações finais acerca do resultados obtidos com o comparativo dos benchmarks e propor melhorias.
            \end{itemize}


\section{Metodologia}

    TODO - Escolher Metodologia de Pesquisa

\section{Organização do Trabalho}

    Esse TCC está organizado em oito capítulos, sendo este primeiro a Introdução. Os
    demais capítulos são resumidos brevemente a seguir:

    \begin{itemize}
        \item \textbf{Revisão sistemática:} Neste capítulo existirão subtópicos explicando alguns conceitos de CDR de forma mais
                                                                    aprofundada, e também os ferramentais como \textit{Hadoop, Yarn, MapReduce, Spark,
                                                                    Impala, Hive} e dentre outras ferramentas caso haja necessidade de inclusão, para o
                                                                    aperfeiçoamento do processamento de dados.
        \item \textbf{Montagem do ambiente:} Realiza uma explicação sobre quais os procedimentos que foram realizados,
                                                                         bem como informações sobre a arquitetura hardware e software utilizada, e
                                                                         como foram realizadas as configurações rede entre nós do cluster alvo. O
                                                                         objetivo é apresentar uma visão da arquitetura que será utilizada no cluster.
        \item \textbf{Benchmarks:} Teremos a apresentação do ferramental responsável tanto pelo monitoramento quanto
                                                       pela captura das métricas que serão utilizadas e armazenadas para questões comparativas.
                                                       Explicação de como serão realizado os benchmarks e as possíveis ferramentas de análise
                                                       dos dados fornecido pelos benchmarks.
        \item \textbf{Replicação do processamento de dados:} Apresentação do estudo sobre implantação dos CDR's no cluster,
                                                                                                   inicio do processo de inserção de CDR's, e realização da replicação
                                                                                                   do processamendo dos CDR's no cluster alvo. Explicação do passo
                                                                                                   a passo sobre como foram realizados esses procedimentos, bem
                                                                                                   como se o processo está semelhante ao que é normalmente
                                                                                                   realizado. Neste capítulo temos também a apresentação da captura
                                                                                                   dos benchmarks para o comparativo que será realizado.
        \item \textbf{Resultados Obtidos: } Aqui teremos o comparativo dos benchmarks, bem como as respostas para as
                                                                  questões de pesquisa o qual foram estabelecidas. O objetivo é averiguar se os
                                                                  dados coletados demonstram se as configurações realizadas foram os suficientes
                                                                  para diminuir o tempo de processamento.
        \item \textbf{Considerações Finais:} Por fim, teremos as considerações finais para averiguar que conclusões podem ser
                                                                     retiradas deste TCC, bem como proposição de futuros trabalhos e melhorias para os
                                                                     próximos trabalhos que seguirem na mesmo contexto deste tema.
        \item \textbf{Referencial Teórico:} Material que foi utilizado para o embasamento teórico deste TCC, como artigos,
                                                                  monografias, periódicos e dentre outros trabalhos que deram suporte as idéias
                                                                  desenvolvidas nele.
    \end{itemize}